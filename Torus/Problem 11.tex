\documentclass{article}
\usepackage[utf8]{inputenc}

\usepackage{amsmath}
\usepackage{mathtools}
\usepackage{amssymb}
\usepackage{parskip}
\usepackage[margin=1in]{geometry}

\title{Derivation of the Volume of a Torus}
\author{Over-Aioli2153}
\date{\today}

\begin{document}
	
	\maketitle
	\section*{Problem}
	Let $0<r<R$. Let $T$ be the solid circular torus ('donut') generated by the circle
	\[\left\{(x,y,0)\in \mathbb{R}^3:\sqrt{\left(x-\frac{R+r}{2}\right)^2+y^2}\leq\frac{R-r}{2}\right\}\]
	(in the X-Y plane) turning around the vertical axis (Y axis). In this case, $R$ is the distance from the axis of revolution (Y axis) to the farthest part of the tours, and $r$ is the distance from the axis of revolution to the nearest part of the torus. Show that $\text{vol}(T)=\frac{\pi^2}{4}(R+r)(R-r)^2$
	\section*{Semicircle Definitions}
	Considering the torus will be generated by rotating a circle with respect to the $y$ axis, and the constants $R$ and $r$ provided in the problem, we can define the equations for the outer semicircle and inner inner circle as:
	
	\begin{align*}
		\text{Outer Semicircle:} & \quad f_e(y) = \sqrt{-y^{2} + \left(\frac{R-r}{2}\right)^{2}} + \frac{R+r}{2} \\
		\text{Inner Semicircle:} & \quad  f_i(y) = -\sqrt{-y^{2} + \left(\frac{R-r}{2}\right)^{2}} + \frac{R+r}{2}
	\end{align*}
	
	\section*{Integral for the Volume of the Torus}
	Given, $\frac{R-r}{r}$ functions as the circle's radius, we can use it's negative as lower bound, and it's positive as uppper bound, to end up with the following integral:
	
	\begin{equation}
		V_t = \pi \int_{-\frac{R-r}{2}}^{\frac{R-r}{2}} \left[ f_e(y)^2 - f_i(y)^2 \right] dy
	\end{equation}
	
	\section*{Simplification}
	We can define the following substitutions to simplify the equation:
	\begin{align*}
		a &= \sqrt{-y^{2} + \left(\frac{R-r}{2}\right)^{2}} \\
		b &= \frac{R+r}{2}
	\end{align*}
	
	And rewrite the functions as follows:
	\begin{align*}
		\text{Outer Semicircle (simplified):} & \quad f_e(y) = a + b \\
		\text{Inner Semicircle (simplified):} & \quad f_i(y) = -a + b
	\end{align*}
	
	\section*{Developing the Integral}
	We can now substitute the functions inside the integral:
	\begin{align*}
		V_t &= \pi \int \left[ (a + b)^2 - (-a + b)^2 \right] dy \\
		&= \pi \int \left[ a^2 + 2ab + b^2 - (a^2 - 2ab + b^2) \right] dy \\
		&= \pi \int 4ab \, dy \\
		&= 4\pi b \int a \, dy \\
		&= 4\pi b \int_{-\frac{R-r}{2}}^{\frac{R-r}{2}} \sqrt{\left(\frac{R-r}{2}\right)^2 - y^2} \, dy
	\end{align*}
	
	\section*{Trigonometric Substitution}
	\subsection*{Trigonometric Simplification}
	To more easily apply trigonometric substitution, we can consider that:
	\[ \beta=\frac{R-r}{2}\]
	Thus:
	\[4\pi b \int_{-\beta}^{\beta} \sqrt{\beta^2 - y^2} \, dy\]
	\subsection*{Substitution}
	We can begin by establishing that the square root is the pythagorean addition of $b_t$, our hypothenuse:
	\[b_t = \sqrt{(\beta^2 - y^2)}\]
	Which then implies that these are the following trigonometric properties of our pythagorean addition:
	\begin{align*}
		\sin{\theta} &= \frac{y}{\beta} \\
		y &=\sin{(\theta)} \beta  \Rightarrow dy =\cos{(\theta)}\beta \space d\theta \\
	\end{align*}
	and
		\[\cos{(\theta)} = \frac{b_t}{\beta}  \Rightarrow b_t = \cos{(\theta)}\beta\] 
	With which we can now sutbitute our current integral to:
	\[4\pi b \beta^2\int_{-\beta}^{\beta} \cos^2{(\theta)}\, d\theta\]
	With integration formulas we can now convert our equation to:
	\[4\pi b \beta^2\left[\frac{\theta}{2}+\frac{\sin{(2\theta)}}{4}\right]_{-\beta}^{\beta}\]
	To substitute $\theta$ into its equivalence in terms of $y$, we can refer back to our trigonometric properties and see that:
	\[\theta = \arcsin{\left(\frac{y}{\beta}\right)}\]
	Which now allows us to see the transformed integral in terms of $y$:
		\[4\pi b \beta^2\left[\frac{	\arcsin{\left(\frac{y}{\beta}\right)}}{2}+\frac{\sin{(2	\arcsin{\left(\frac{y}{\beta}\right)})}}{4}\right]_{-\beta}^{\beta}\]
	The limits of the integral can now be evaluated, for simplicity, the trigonometric functions will be evaluated separately for $\pm \beta$ and then replaced onto their respective places within the fully evaluated equation:
	\begin{align*}
		\text{For $\beta$:}\\
		\arcsin{\left(\frac{\beta}{\beta}\right)} &= \arcsin{1}=\frac{\pi}{2}\\
		\sin{\left(2\arcsin{\left(\frac{\beta}{\beta}\right)}\right)}&=\sin{(\pi)}=0\\
	\end{align*}
	\begin{align*}
		\text{For $-\beta$:}\\
		\arcsin{\left(\frac{-\beta}{\beta}\right)} &= \arcsin{-1}=-\frac{\pi}{2}\\
		\sin{\left(2\arcsin{\left(\frac{-\beta}{\beta}\right)}\right)}&=\sin{(-\pi)}=0\\
	\end{align*}
	Now we can place them onto the fully evaluated equation: 
	\begin{align*}
			4\pi b \beta^2\left[\frac{\frac{\pi}{2}}{2}-\frac{-\frac{\pi}{2}}{2}\right]=&4\pi b \beta^2\frac{\pi}{2} \\
			=&2\pi^2 b \beta^2
	\end{align*}
	The equation is now only in terms of  $b$ and $\beta$, we can recall their definitions and substitute them back to finalize:
	\begin{align*}
		b &= \frac{R+r}{2} \\
		\beta &=\frac{R-r}{2} \\
		&\Rightarrow 2\pi^2 \left(\frac{R+r}{2}\right) {\left(\frac{R-r}{2}\right)}^2\\
		&=\frac{\pi^2}{4}(R+r)(R-r)^2
	\end{align*}
	
	\[
	\boxed{V_t = \frac{\pi^2}{4}(R+r)(R-r)^2}
	\]
	
\end{document}